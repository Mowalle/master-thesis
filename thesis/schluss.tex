\chapter{Fazit und Ausblick}
\label{chap:closing}
Die digitale Unterstützung der Navigation und Exploration durch augmentierte Inhalte ist ein Thema, das in der Vergangenheit bereits ausgiebig von Forschung und Industrie behandelt wurde.
Die Idee, eine augmentierte Karte als Megamap in die Umgebung der Nutzer zu integrieren, ist jedoch bisher nur aus Videospielen wie z.B. Tom Clancy's The Division bekannt.
In dieser Masterarbeit wurde eingehend untersucht, ob sich eine solche Megamap auch für die \enquote{reale} Welt umsetzen lässt.
Zu diesem Zweck wurde das Konzept einer Megamap für den Einsatz mit einem Mixed-Reality-HMD entwickelt.
Mit der Megamap sollen virtuelle Gebäudekarten in das reale Umfeld der Nutzer projiziert werden, sodass diese sich einen Überblick der Umgebung verschaffen können.
Verschiedene explorative Interaktionselemente aus bereits existierenden Kartenanwendungen sind in das Konzept mit eingeflossen und wurden für den Einsatz in Innenbereichen neu interpretiert.
Auf Basis dieses Konzepts wurde ein erster Prototyp der Megamap für die HTC~Vive entwickelt.
In einer Nutzerstudie ($N=15$) wurde mit dem Prototyp die Nutzbarkeit, Effektivität und Effizienz der Megamap für die Suche nach Objekten und den Aufbau einer räumlichen Vorstellung der Umgebung untersucht.
Die Megamap wurde in zwei Varianten getestet (Fußboden- und Bauchhöhe).
Als Vergleichsbasis diente hierzu eine herkömmliche 2D-Darstellung der Gebäudekarten.
Die statistische Analyse der quantitativ erhobenen Daten ergab keine Vorteile der Megamap gegenüber der 2D-Variante in Bezug auf Effektivität oder Effizienz.
Weiterhin bewerteten die Probanden die Nutzbarkeit der 2D-Alternative besser als die der Megamap.
In einem kurzen Leitfadeninterview wurden zudem die subjektiven Meinungen der Probanden zu der Megamap eingeholt.

\section{Offene Fragen und Probleme dieser Arbeit}

\section{Potential der AR-Indoor-Navigation und -Exploration}
% Technische Neuerungen im Bereich AR (und Anderes)
% ==> Tellerrand!

%
\cleardoublepage
