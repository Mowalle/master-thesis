\chapter*{\abstractname}
\thispagestyle{empty}

Diese Masterarbeit widmet sich dem Thema, wie die Erkundung von Gebäuden mithilfe von digital augmentierten Innenraumkarten unterstützt werden kann.
Hierfür wird das Konzept der \emph{Megamap} präsentiert.
Es handelt sich dabei um eine neuartige Form, dreidimensionale Gebäudekarten mithilfe eines Mixed-Reality Head-Mounted-Displays darzustellen.
Die virtuellen Karten werden in die reale Umgebung der Nutzer integriert, sodass diese inmitten der Karte stehen.
Über augmentierte Interaktionselemente und durch die Bewegung auf der Karte können die Nutzer ihre Umgebung explorativ erforschen.
Um die Realisierbarkeit dieses Konzepts nachzuweisen wurde ein erster Prototyp der Megamap für die HTC~Vive implementiert und in einer Nutzerstudie quantitativ und qualitativ getestet.
Zwei Varianten der dreidimensionalen Megamap (Fußboden- und Bauchhöhe) wurden einer konventionellen, zweidimensionalen Darstellung von Gebäudekarten gegenübergestellt.
Der Prototyp wurde auf seine Eignung zur Suche nach Objekten sowie den Aufbau eines mentalen Modells der Raumanordnung getestet.
Zu diesem Zweck führten die Probanden eine Suchaufgabe und eine Richtungsschätzung auf unterschiedlichen Gebäudekarten durch.
Die Effektivität und Effizienz der jeweiligen Variante wurde anhand der Suchzeit, der Präzision der Schätzung und der Nutzbarkeit gemessen.
Es zeigte sich, dass mit der 2D-Darstellung die Suche nach den Objekten \SIrange{33}{36}{\percent} schneller war als mit den Megamaps.
Für die Richtungsschätzung ergaben sich keine statisch signifikanten Unterschiede.
Dementsprechend schätzten die Probanden die 2D-Darstellung bezüglich der Nutzbarkeit als die präferierte Variante ein.
Aus den Ergebnissen und dem Feedback der Probanden wird geschlossen, dass der Prototyp das Konzept der Megamap unzureichend umsetzt und das Potential einer augmentierten, umgebungsintegrierten Gebäudekarte nicht ausschöpft.
Für eine zufriedenstellende Implementierung der Megamap ist die weitere Entwicklung, sowohl des Prototypen als auch der Mixed-Reality-Technologie im Allgemeinen, notwendig.

\cleardoublepage

\begin{otherlanguage}{british}
\chapter*{\abstractname}
\thispagestyle{empty}

This master's thesis presents research about how the exploration of buildings can be supported by the means of digitally augmented indoor maps.
Based on previous work and applications, a novel approach to displaying three-dimensional indoor maps via mixed-reality head-mounted-displays is proposed: the \emph{Megamap}.
The virtual maps are integrated into the users' surroundings, such that the maps are centred around them.
The users can explore their environments by using interactive augmented elements as well as moving physically on the map.
To proof the feasibility of the concept, a prototypical implementation of the Megamap was created for the HTC~Vive, which was evaluated in a quantitative and qualitative user study.
Two variations of the three-dimensional Megamap (ground and chest height) were tested against a conventional two-dimensional indoor map display.
The study's purpose was to find out whether the prototype is suitable for searching objects on the map and for creating a mental representation of the rooms' layout.
To this end, the participants were tasked to complete a search and a direction estimation task on varying indoor map layouts.
The effectiveness and efficiency of each variation was measured by the search time, precision of the estimation as well as the usability.
While the 2D variant was \SIrange{33}{36}{\percent} faster for the search task than the Megamaps, there was no statistically significant difference in terms of direction estimation precision between the conditions.
For that reason, the usability ratings given by the participants show a clear preference towards the 2D map display.
Based on these results and the participants' feedback, it is concluded that the prototype does not appropriately reflect the Megamap concept and misses the potential of an augmented, environmentally integrated indoor map.
A proper implementation of the Megamap concept requires further development, both of the prototype and mixed-reality technology in general.

\end{otherlanguage}
\cleardoublepage
