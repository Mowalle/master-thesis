\chapter{Konzeption der Megamap-Anwendung}
\label{chap:concept}
Dieses Kapitel beschreibt die Konzeption der im Rahmen dieser Arbeit entwickelten Megamap-Anwendung.
Hierfür werden zunächst mögliche Anwendungsszenarien präsentiert und die sich daraus ergebenden Anforderungen genannt.
Wie bereits erwähnt wird die Kartenexploration verstärkt fokussiert als die Navigation.
Zu diesem Zweck werden existierende Ansätze zur digitalen Kartenexploration analysiert und die Gemeinsamkeiten der Ansätze als Grundlage für die Megamap-Anwendung herausgearbeitet.
%Schließlich wird ein Überblick der für die Implementierung benötigten Komponenten gegeben.

\section{Anwendungsszenario und Anforderungen}
\label{sec:reichenbacher_use_cases}
% Hier die Guidelines einarbeiten
Bereits \textcite{Reichenbacher2001} beschreibt Konzepte zur mobilen Nutzung von digitalen Kartensystemen.
Unter anderem gibt er eine Übersicht verschiedener \emph{User Tasks}, die in mobilen Umgebungen ausgeführt werden können (siehe \autoref{tab:gis_user_tasks}).
\todo{Was heißt denn \enquote{kontextbasiert}?}
Neben der \textbf{Navigation} finden sich auch kontextbasierte Tasks wieder.
\citeauthor{Reichenbacher2001} nennt drei Kategorien kontextbasierter Tasks:

\textbf{Lokalisierungs-Tasks (\emph{Locators})} sind Anwendungsfälle, bei denen Nutzer Positionen abfragen und diese angezeigt werden.
Dabei kann es sich um die eigene Position handeln, aber auch um die von anderen Personen oder Objekten.

\textbf{Nähe-Tasks (\emph{Proximity})} sind Anwendungsfälle, bei denen Nutzer Informationen über die Umgebung relativ zur eigenen Position erhalten.
So können Nutzer z.B. die Umgebung nach den nächstgelegenen Geschäften oder Freizeitbeschäftigungen abfragen.

\textbf{Events} sind nicht nur ortsabhängig, sondern auch zeitabhängig.
Bei diesen Anwendungsfällen geht es um die Abfrage von Informationen, die sich mit der Zeit ändern.
Z.B. fallen die Abfrage von geöffneten Geschäften oder von Verkehrsinformationen in diese Kategorie.

% TODO: Definition Exploration?
Für diese Arbeit werden die oben genannten kontextbasierten Geoinformationsaufgaben als Anwendungsfälle der Kartenexploration betrachtet.
Konkretere Beispiele für Anwendungsfälle werden in den nächsten Abschnitten bei der Betrachtung existierender Systeme deutlich.

\begin{table}[tbh]
    \centering
    \caption{Geoinformationsaufgaben in einer mobilen Umgebung. \quelle{\cite[47]{Reichenbacher2001}}}
    \label{tab:gis_user_tasks}
    \begin{tabular}{@{}lll@{}}\toprule
        \textsf{\textbf{Tasks}} & \textsf{\textbf{Subtasks}} & \textsf{\textbf{Examples}}\\ \midrule
        \multirow{3}{*}{Locators} & Own position & xy coordinates, place name \\
                                  & Objects & Attributes of an object\\
                                  & Other persons position & Who is this?\\ \midrule
        \multirow{2}{*}{Proximity} & Objects & Next object with certain attributes\\
                                   & Persons & Known people in the area?\\ \midrule
        Navigation & Routing & Way descriptions\\ \midrule
        Events & What happens at a place & Obstacles (e.g. traffic jam)?\\ \bottomrule
    \end{tabular}
    \vspace{0.5em}
\end{table}

\section{Kartenexploration in existierenden Anwendungen}
Bevor ein Konzept für die Megamap formuliert wird, werden Darstellungselemente und Interaktionsmöglichkeiten zur Kartenexploration in bereits existierenden Anwendungen untersucht.
Die Idee ist, diejenigen Elemente in der Megamap zu übernehmen, welche Nutzern bereits aus den anderen Anwendungen bekannt sind.
Durch die Verwendung der bekannten Elemente und Interaktionen soll ein zusätzlicher Lernaufwand für die neuartige Megamap-Anwendung verhindert werden.

Als Beispiele für existierende Kartenanwendungen werden in dieser Arbeit die \emph{Desktop}-Varianten von \emph{Google Maps}, \emph{Bing Maps} \parencite{Microsoft2018b} und \emph{HERE WeGo} \parencite{HERE2018} untersucht.
Zur Kategorisierung der gefundenen Elemente werden diese den zuvor beschriebenen Anwendungsfällen Lokalisierung, Nähe und Event zugeordnet (siehe \autoref{sec:reichenbacher_use_cases}).

\subsection{Kartenexploration mit Google Maps}
\subsection{Kartenexploration mit Bing Maps}
\subsection{Kartenexploration mit HERE WeGo}

\section{Megamap in \emph{Tom Clancy's The Division}}
% Mechanics

\section{Anwendung der Explorationselemente für Megamap}
\begin{table}[tbh]
    \small
    \centering
    \caption{Übersicht der Explorationselemente in ausgewählten Anwendungen}
    \label{tab:exploration_elements_summary}
    \begin{tabular}{@{}lcccc@{}}\toprule
        \textsf{\textbf{Explorationselement}} & \textsf{\textbf{Google Maps}} & \textsf{\textbf{Bing Maps}} & \textsf{\textbf{Here WeGo}} & \textsf{\textbf{TCTD}}\\ \midrule

        \multicolumn{5}{@{}l@{}}{\textsc{Lokalisierung}} \\ \midrule
        Positionsmarker (Nutzer) & \checkmark & \checkmark & \checkmark & \\
        Labels (Straßen, Flüsse, \dots) & \checkmark & \checkmark & \checkmark & \\
        Gebäudemarkierungen & \checkmark & \checkmark & \checkmark & \\
        Klickmarkierung (Adresse, Koordinaten) & \checkmark & \checkmark & \checkmark & \\
        Spezifische Suche (mittels Suchfeld) & \checkmark & \checkmark & \checkmark & \\

        \midrule
        \multicolumn{5}{@{}l@{}}{\textsc{Nähe}} \\ \midrule
        Gebäudemarkierungen (Positionsabh.) & \checkmark & \checkmark & \checkmark & \\
        Gebäudemarkierungen (Zoomabh.) & \checkmark & \checkmark & \checkmark & \\
        Entfernungen (außerhalb Navigation) & \checkmark & \checkmark & --- & \\
        Terrain & \checkmark & --- & --- & \\
        Nahbereichssuche & \checkmark & \checkmark & --- & \\
        Filter u. Sortierung & \checkmark & \checkmark & --- & \\
        Offene Suche (mittels Suchfeld) & \checkmark & \checkmark & \checkmark & \\
        Gebäudenachbarschaft & \checkmark & \checkmark & Nur Transport & \\
        Ausstattung (z.B. Hotels etc.)  & \checkmark & \checkmark & --- & \\
        Umgebungsbilder & \checkmark & \checkmark & --- & \\

        \midrule
        \multicolumn{5}{@{}l@{}}{\textsc{Event}} \\ \midrule
        Öffnungszeiten & \checkmark & \checkmark & \checkmark & \\
        Verkehrsinformationen & \checkmark & \checkmark & \checkmark & \\
        Veranstaltungen & \checkmark & --- & --- & \\
        Besucheraufkommen & \checkmark & --- & --- & \\

        \bottomrule
    \end{tabular}
\end{table}
%
\cleardoublepage
