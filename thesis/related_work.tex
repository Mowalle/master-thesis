\chapter{Stand der Forschung}
\label{chap:related_work}
Ein großer Teil der Forschung setzt sich mit \acro{AR}-unterstützter Navigation beim Autofahren auseinander, da hier eine fehlerfreie Navigation wichtig ist und im Fehlerfall sogar fatale Folgen haben kann \parencite{Lin2017}.
Ein häufig eingesetztes Mittel ist hier die virtuelle Hervorhebung der Straßenführung.
\textcite{Narzt2006} präsentieren ein allgemeines Framework zur Darstellung einer virtuellen Route.
Das Framwork kann sowohl für Fahrzeug- als auch Fußgängernavigation und auf verschiedenen Endgeräten eingesetzt werden.
Für den Einsatz im Auto schlagen die Autoren die Windschutzscheibe als Display vor.
Dadurch wäre das AR-Display in der natürlichen Umgebung des Nutzers integriert.

\textcite{Bark2014} erweitern diesen Ansatz.
Sie ersetzen die statische virtuelle Route durch eine Reihe animierter Papierflieger, die entlang der geplanten Route fliegen in einem zuvor aufgenommenen Video.
Ein Vergleich mit einer herkömmlichen \acro{GPS}-Top-Down-Darstellung zeigt, dass die Probanden dank der \acro{AR}-Darstellung dem Straßenverlauf besser folgen können.
Allerdings verliert die \acro{AR}-Darstellung durch das zweidimensionale Display ihre Effektivität.
Die Autoren betonen daher die Notwendigkeit eines dreidimensionalen Volumendisplays.

\textcite{Kim2009} erweitern die Route durch eine \acro{2.5D}-Karte mithilfe eines Fahrsimulators.
Von der Oberseite der Windschutzscheibe schiebt sich eine 2D-Karte ins Sichtfeld, die sich nach und nach dreidimensional mit der Straße verbindet.
Beim Vergleich mit einem konventionellen \acro{GPS}-Navigationsdisplay zeigt sich, dass die Probanden weniger Fehler bei der Navigation machen und den Straßenverkehr aufmerksamer verfolgen können.

\cite{Wiesner2017} versuchen, die Ungenauigkeit von \acro{GPS}-Systemen beim Platzieren von virtuellen Objekten in der Umgebung zu umgehen.
Dafür wählen sie eine Pfeil-Darstellung, die mit dem Koordinatensystem des Autos registriert ist und nur von der Entfernung zur nächsten Kreuzung abhängt.
% TODO Reultate?

Neben der Navigation im Auto ist auch die Fußgängernavigation ein weit untersuchtes Thema in der Forschung.
Bereits \textcite{Hoellerer1999} stellen ein \acro{AR}-System vor, bei dem zwei Nutzer kooperieren können.
Der eine Nutzer läuft mit einem \acro{HMD} und einem Tablet-Computer in der Außenwelt herum.
Durch das \acro{HMD} können ihm augmentierte Informationen zur Umgebung und sogar ganze virtuelle Gebäude dargestellt werden, die mit der Umgebung registriert sind.
Der andere Nutzer kann entweder über ein Desktop-Interface oder eine \acro{AR}-Darstellung der Karte Routen und Informationen in der Umgebung platzieren.

\textcite{Wen2014} gehen von der Annahme aus, dass Menschen durch verbesserte Navigationssysteme das Verständnis für die eigene Umgebung und die Fähigkeit, selbstständig zu navigieren, verlieren.
Sie stellen ein \acro{AR}-Navigationssystem vor, dass erst dann die zu laufende Route anzeigt, wenn ein Bilderrätsel zu einem Gebäude in der Umgebung beantwortet wurde.
Probanden des Systems reagierten gemischt.
Während die einen das System als zu umständlich empfanden, konnten die anderen ein besseres mentales Bild von der abgelaufenen Umgebung bestätigen.

\textcite{Alnabhan2014} präsentieren einen Ansatz zur \emph{Indoor}-Navigation als \emph{Android-App}.
Die App nutzt \emph{\acro{Wifi}-Fingerprinting} und den digitalen Kompass eines Smartphones/Tablets zur Positionierung und Orientierung.
Über vordefinierte Referenzpunkte im Gebäude wird die Ungenauigkeit des \acro{Wifi}-Fingerprintings ausgeglichen.
Navigationshinweise werden als \acro{AR}-Pfeil angezeigt (mit zusätzlichen Textausgaben).
Alternativ dazu zeigen \textcite{Liu2016} einen ähnlichen Ansatz, bei dem die Positionierung jedoch auf Magnetfeldsensoren aufbaut und daher weder auf \acro{GPS} noch auf Infrastruktur wie \acro{Wifi} oder Marker angewiesen ist.
\textcite{Rehman2017} zeigen \acro{AR}-Pfeile und Textinformation per \emph{Google Glass}.
Ihr Vergleich mit der gleichen Anwendung auf einem Smartphone zeigt, dass die wahrgenommene Genauigkeit bei dem tragbarem Google Glass höher ist. Jedoch lässt sich die Umgebung in beiden digitalen Varianten schlechter einprägen als mit einer Papierkarte.

% WIM
Ein weiterer Ansatz zur Unterstützung von Navigation und Interaktion, der vor allem in virtuellen Welten Anwendung findet, ist die \emph{\acro{World-in-Miniature}}-Metapher (WIM) \parencite{Stoakley1995}.
Dabei wird die nähere Umgebung miniaturisiert dargestellt und häufig als eigenständiges Objekt vom Nutzer in der Hand gehalten.
Je nach Anwendung ergeben sich hieraus unterschiedliche Interaktionsmöglichkeiten.

\textcites{Mulloni2011a}{Mulloni2012} nutzen einen kombinierten Ansatz aus dem zuvor beschriebenen \acro{AR}-Pfeilen und WIM.
Während Nutzer durch ein Gebäude navigieren werden \acro{AR}-Pfeile sowie Navigationsanweisungen in Form von \enquote{Aktivitäten} angezeigt.
Sobald zuvor platzierte Poster an Infopunkten im Gebäude erreicht werden, wechselt die Ansicht zu einer WIM-Darstellung der Etage mit der hervorgehobenen Route zum Zielort.
Laut dem Ergebnis der Studie der Autoren hilft die WIM-Übersicht den Nutzern, gewisse Orientierungspunkte (\emph{Landmarks}) in der Umgebung wiederzuerkennen, was zu weniger Fehlern bei der Navigation führt.

\textcite{Chittaro2005} erweitern das Konzept von WIM für mehrstöckige Gebäude als eine \emph{BreakAway}-Karte.
Hier können einzelne Stockwerke zur Seite geschoben und ausgeblendet werden, um so auf Stockwerke blicken zu können, auf denen sich der Nutzer selbst gar nicht befindet.

\textcite{Vallance2001} schlägt zudem eine Biegung der WIM-Umgebung vor, um so eine Verdeckung durch Wände und andere Objekte zu minimieren.

Alle zuvor genannten Arbeiten haben gemeinsam, dass die untersuchte Hautpaufgabe die Navigation von einem Punkt zu einem anderen ist.
Allerdings gibt es auch Anwendungen, in denen Navigation zwar eine Rolle spielt, jedoch die Start- und Zielpunkte nicht klar vorgegeben sind.
Wenn z.B. in der Umgebung nach Kontext gesucht wird (beispielsweise alle Restaurants in einem gewissen Umkreis) spricht man von \emph{Exploration} \parencite[28]{Lodts2015}.

Ein Bereich, in dem sich Navigation und Exploration die Waage halten, sind digitale Spiele.
Einerseits müssen Spieler in einem gewissen Rahmen zum Ziel geführt werden, da sie ansonsten frustriert sind und das Spiel beenden.
Andererseits darf durch Navigationshinweise das Spiel nicht zu einfach gestaltet werden, damit das Spiel immer noch genug Herausforderungen bietet und den Spielern die Freiheit der Entdeckung erhalten bleibt.
% TODO Übergang?
\textcites{Moura2014}{Moura2015} untersuchen in einer breiten Auswahl an Spielen den Zusammenhang zwischen auftretenden Navigationshelfern, Design der Level sowie Spielemechaniken zur Unterstützung der Navigation.
Daraus leiten \textcite{Moura2015} eine Klassifikation von Navigationshelfern ab.
Außerdem erstellen sie eine Liste von Design-Richtlinien, welche die Vor- und Nachteile der jeweiligen Helfer erläutern und als Hilfestellung für Entwicklung von neuen Spielen dienen.

Aufbauend auf die Navigationshelfer in Spielen untersucht \textcite{Lodts2015}, ob diese Helfer auch in der realen Welt anwenbar sind.
Dazu wurden unterschiedliche Helfer, wie sie aus Spielen bekannt sind, für das Google Glass implementiert.
In einem Test, aufgeteilt in die Aufgabenbereiche Navigation, Orientierung, Exploration und Annotation, wurde ermittelt welche der Helfer den Nutzern bei der jeweiligen Aufgabe am hilfreichsten waren.
Es zeigt sich, dass Helfer, die in der Umgebung eingebettet sind (z.B. \acro{AR}-Pfeile oder -Text, farbliche Hervorhebungen von Objekten) effektiver sind als statische Anzeigen als \acro{\emph{Head-Up Display}(HUD)}.

Zur \acro{AR}-Exploration tragen \textcite{Reitmayr2005} bei, indem sie eine reale Papierkarte durch einen Projektor augmentieren.
Ein projizierter Helikopter kann mit einem realen \acro{PDA} bewegt werden, wenn dieser in die Nähe gehalten wird.
Der \acro{PDA} zeigt währenddessen ein Interface zur Kontrolle des Helikopters.
Der Helikopter dient als Cursor für kontextbasierte Informationen.
Sobald der Nutzer einen Rahmen auf die Karte legt, wird auf den Rahmen ein Bild der aktuellen Kartenposition aus ungefährer Sicht des Helikopters projiziert.
Zudem kann der Zustand der Karte durch die Projektion dynamisch verändert werden (in diesem Fall werden Gezeiten von Gewässern simuliert).

%
\cleardoublepage