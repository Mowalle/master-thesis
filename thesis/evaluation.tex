\chapter{Nutzerstudie}
\label{chap:evaluation}

\section{Aufbau}
Das Experiment wurde im Laborraum 5220 (MZH Ebene 5) der Arbeitsgruppe Human-Computer Interaction an der Universität Bremen durchgeführt.
Für die Präsentation der virtuellen Umgebung wurde die HTC Vive mit zwei Basisstationen eingesetzt.
Der getrackte Bereich zwischen den Basisstationen umfasste ca. \SIrange{X}{Y}{\metre}.
Die Bilddaten wurden von einem PC mit einer \emph{NVidia~GTX~1080~Ti} Grafikkarte und einem \emph{AMD~Ryzen~7~1800X} Prozessor gerendert.
Übertragen wurden die Bilddaten an das HMD mit dem \emph{VIVE~Wireless~Adapter} \parencite{HTCCorporation2018b}.
Die Fragebögen wurden über \emph{Google Forms} erstellt und ausgefüllt.
Hierfür wurde ein separater Laptop bereitgestellt.

\section{Konditionen und Aufgaben}

\section{Ablauf}

\section{Ergebnisse}

\section{Zusammenfassung und Diskussion der Ergebnisse}

%
\cleardoublepage
