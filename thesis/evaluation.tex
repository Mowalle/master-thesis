\chapter{Nutzerstudie}
\label{chap:evaluation}
\todo{Fragebögen, Flyer, Infobögen etc. in Anhang.}
Wie bereits in \autoref{chap:related_work} erwähnt ist ein Nachteil von 2D-Karten, dass durch das Fehlen einer Dimension zum Abgleich mit der Umgebung ein gewisser mentaler Aufwand notwendig ist.
Die 2D-Karte stellt eine Abstraktion der Umgebung dar, wodurch Informationen verloren gehen.

Der Zweck dieser Nutzerstudie ist zu überprüfen, ob sich bei den Probanden durch die 3D-Megamap ein räumliches Verständnis aufbaut, welches gegenüber der 2D-Karte zu einer performanteren und effizienteren Orientierung führt.
Unter Performanz und Effizienz wird in diesem Sinne verstanden, dass die Fehlerquote und die Zeit zum Einschätzen von Richtungen zu Orten in der Umgebung niedriger sind, als bei der 2D-Variante.
Zudem sollen die subjektiven Eindrücke und Präferenzen der Probanden von den unterschiedlichen Kartenvarianten ermittelt werden.
Für zukünftige Arbeiten werden außerdem Verbesserungsvorschläge und Ansichten über das Potential der Megamap gesammelt.

\section{Aufbau}
Das Experiment wurde im Laborraum 5220 (MZH Ebene 5) der Arbeitsgruppe Human-Computer Interaction an der Universität Bremen durchgeführt.
Für die Präsentation der virtuellen Umgebung wurde die HTC Vive mit zwei Basisstationen eingesetzt.
Der getrackte Bereich zwischen den Basisstationen umfasste ca. \SIrange{0}{0}{\metre}.\todo{Tatsächliche Zahlen einsetzen.}
Die Bilddaten wurden von einem PC mit einer \emph{NVidia~GTX~1080~Ti} Grafikkarte und einem \emph{AMD~Ryzen~7~1800X} Prozessor gerendert.
Übertragen wurden die Bilddaten an das HMD mit dem \emph{VIVE~Wireless~Adapter} \parencite{HTCCorporation2018b}.
Die Fragebögen wurden über \emph{Google Forms} erstellt und ausgefüllt (siehe \autoref{appendix:study_material}).
Hierfür wurde ein separater Laptop bereitgestellt.

\section{Konditionen und Aufgaben}
\label{sec:conditions_and_tasks}
Für die Nutzerstudie wurden drei unterschiedliche Konditionen getestet:
\begin{itemize}
    \item $3D_l$ (\emph{low)}: Die 3D-Megamap wird mit einer Skalierung von \SI{6}{\percent} (relativ zur Umgebung) und \SI{25}{\cm} über dem Boden angezeigt.
    Diese Kondition entspricht am ehesten der Megamap aus TCTD. 
    \item $3D_h$ (\emph{high}): Die 3D-Megamap wird mit einer Skalierung von \SI{6}{\percent} (relativ zur Umgebung) und \SI{1}{\metre} unterhalb der HMD-Position angezeigt.
    Dabei wird die Höhe nur beim Aufrufen der Karte berechnet.
    Danach bleibt die Höhe fix.
    Diese Kondition wird getestet, um zu ermitteln, wie sich die Höhe der Megamap vom Boden auf die Nutzung auswirkt (insbesondere unter dem Aspekt, dass durch die Darstellung in 3D Verdeckungen durch die Wände der Räume auftreten können).
    \item $2D$: Eine 2D-Karte wird senkrecht an der Wand vor dem Nutzer angezeigt.
    Auch die 2D-Karte hat eine Skalierung von \SI{6}{\percent} (relativ zur Umgebung).
    Die Karte ist so an der Wand platziert, dass sie nicht von anderen Wänden oder der Decke und dem Fußboden geschnitten wird.
\end{itemize}
\todo{Screenshots der Konditionen einfügen.}

Jede Kondition wurde über 6~Iterationen (\emph{Tasks}) $T_0, \dots, T_5$ wiederholt.
\todo{Bild für Megamap Varianten} Dabei wurde in jeder Iteration eine andere Indoor-Karte für die Megamap verwendet, um einen Lerneffekt für die Kartenlayouts zu minimieren.
Damit die Konditionen vergleichbar bleiben, wurde jede Karte einmal jeder Kondition unterzogen.
Das heißt, die Probanden durchliefen insgesamt 18~Iterationen, wobei sechs unterschiedliche Karten jeweils dreimal eingesetzt wurden.
Die Reihenfolge der Iterationen innerhalb einer Kondition war zufällig bestimmt, wobei darauf geachtet wurde, dass bei einem Wechsel der Konditionen die gleiche Karte nicht zweimal aufeinander folgte.

Um die Auswirkungen der jeweiligen Kondition auf die mentale räumliche Vorstellung des Probanden von der Umgebung zu testen, wurden eine Suchaufgabe und eine Richtungsschätzung in den Prototypen integriert.

Für die Suchaufgabe wurden auf der jeweiligen Megamap in sieben räumen Bälle platziert.
Die Probanden mussten dann den Raum suchen, der die meisten Bälle enthält und diesen mit dem VIVE Controller und dem virtuellen Laserpointer auswählen.
Jede Megamap hatte genau einen Zielraum, welcher zufällig zwischen sieben und zehn Bällen enthielt.
Die anderen Räume enthielten mindestens fünf Bälle und garantiert weniger als der Zielraum.
Jede der sechs Indoor-Karten hatte denselben Zielraum in allen drei Konditionen.
Dies wurde entschieden, damit die Karten über Konditionen hinweg vergleichbar bleiben.
Wäre immer ein anderer Zielraum gewählt worden, wäre der Schwierigkeitsgrad des Findens zwischen den Konditionen unter Umständen verschieden gewesen, obwohl die Karte die gleiche wäre.

Für den Fall, dass ein Proband einen falschen Raum auswählt, würde dieser rot eingefärbt werden.
Der Proband müsste dann die Suche nach dem Zielraum fortfahren.
Dieser Fall trat über alle Probanden hinweg außerhalb des Tutorials nur ein einziges Mal auf.

Die Probanden mussten sich anhand der Megamap die Richtung vom virtuellen Laborraum zum Zielraum (horizontal und vertikal zentriert) merken.
Sobald sie den Zielraum auswählten, wurde die Megamap ausgeblendet.
Mit dem virtuellen Laserpointer sollten die Probanden dann möglichst mittig auf den Zielraum \emph{in ihrer Umgebung} zeigen.
Durch Betätigen des Triggers konnte der Laserpointer \enquote{eingefroren} werden, um die Richtung entweder zu korrigieren oder zu akzeptieren.
Wenn die Richtung akzeptiert wurde, startete die nächste Iteration.

Damit die Probanden beim Suchen auf einer Karte immer die gleiche Startposition und Blickrichtung hatten wie in den anderen Konditionen, wurde vor der Suchaufgabe ein \emph{User Setup} implementiert.
Die Probanden mussten sich dabei auf eine festgelegte Zielposition stellen und für zwei Sekunden auf ein Ziel an der Wand schauen.
Die beiden Zielpositionen waren von der jeweiligen Karte abhängig, bleiben jedoch für eine Karte über die Konditionen hinweg gleich.
So wurde sichergestellt, dass die Ausgangsbedingungen für die Suche auf der Karte in jeder Kondition gleich bleiben.

Ebenso wurde das User Setup zwischen der Suche und der Richtungsschätzung implementiert.
Ohne diesen Zwischenschritt hätten die Probanden in den 3D-Konditionen einfach den Laserpointer vom Raum auf der Karte anheben müssen, um die korrekte Richtung zu schätzen.
Durch das User Setup wurde eine Ablenkung erzeugt, welche diesen Vorteil der 3D-Konditionen gegenüber der 2D-Kondition minimieren soll.

\section{Ablauf}
Für einen Test wurden \num{45}~Minuten angesetzt.

Zuerst bekamen die Probanden Hintergrundinformationen zu der Nutzerstudie.
Unter anderem wurde erklärt, dass eine neuartige 3D-Darstellung von Indoor-Karten für MR/VR getestet wird und dass verschiedene Varianten getestet würden.
Mit einem Informationsbogen wurden die Probanden über die Details, den Ablauf sowie die gesammelten Daten der Nutzerstudie aufgeklärt.
Daraufhin unterschrieben die Probanden einen Zustimmungsbogen (siehe \autoref{appendix:study_material}).
Dabei willigten zur Teilnahme an der Studie sowie der Aufzeichnung ihres Blickfelds in der virtuellen Umgebung ein.
Die Probanden konnten frei entscheiden, ob zusätzlich eine Audioaufnahme der Gespräche während des Experiments aufgezeichnet wird.

Vor dem eigentlichen Test füllten die Probanden einen allgemeinen Fragebogen aus (siehe \autoref{appendix:study_material}).
Dieser enthielt Fragen zur Person (Alter, Geschlecht usw.) sowie Vorerfahrungen in VR/MR und mit Kartenanwendungen für Außen- und Innenbereiche. Der letzte Teil des Fragebogens ist die ins Deutsche übersetzte Santa~Barbara~Sense-of-Direction~Skala \parencite{Hegarty2002}, mit dem Probanden Selbstauskunft über ihren Orientierungssinn geben.
Die Bewertung wurde von einer 7-Punkte- auf eine 5-Punkte-Likert-Skala abgeändert, um mit den späteren Fragebögen einheitlich zu sein.

Den Probanden wurde danach das VR-Equipment (HMD und Batteriepack für den Wireless~Adapter) angelegt.
Sie fanden sich im virtuellen Laborraum wieder.
Die Probanden wurden kurz in der Bedienung des VIVE Controllers und der Bewegung im Raum innerhalb der Play Area unterrichtet.
Sie wurden darauf hingewiesen, dass an einer der Wände ein Bildschirm hängt, der ihnen den nächsten Aufgabenschritt in Textform anzeigt, sollten sie den Ablauf vergessen.
Die Probanden konnten auch jederzeit während des Experiments Fragen an den Versuchsleiter stellen.

Es folgte ein Tutorial, bei dem die Probanden die zuvor beschriebenen Aufgaben (User~Setup --- Suche --- User~Setup --- Zeigen) in einer Tutorialkondition durchliefen (\SI{40}{\cm} Kartenhöhe vom Boden bei \SI{4}{\percent} Skalierung).
Zuerst wurde die 3D-Megamap gezeigt, danach die 2D-Kondition.
Die Probanden konnten entscheiden, ob sie das Tutorial wiederholen oder fortfahren wollen.
Zu diesem Zeitpunkt wurde die Bildschirmaufnahme (und ggf. die Audioaufnahme) gestartet.

Nach dem Tutorial durchliefen die Probanden die Konditionen und Aufgaben, wie sie in \autoref{sec:conditions_and_tasks} beschrieben sind.
Nach jeder Kondition setzten die Probanden das HMD ab und füllten am Laptop einen Fragebogen zur zuletzt gestesteten Kartenvariante aus (siehe \autoref{appendix:study_material}).
Die Fragebögen basieren auf der \emph{System~Usability~Scale} \autocite{Brooke2013}, wurden jedoch für das Anwendungsgebiet von Karten angepasst.
Die Bewertung der einzelnen Aussagen erfolgte über eine 5-Punkte-Likert-Skala.

Nach Abschluss des letzten Fragebogens wurde mit den Probanden ein kurzes (5--10 Minuten) Leitfadeninterview geführt.
Unter anderem beantworteten die Probanden Fragen zu ihrer präferierten Kartenvariante fürs Suchen und Richtungsschätzen, ihr Vorgehen beim Suchen sowie Verbesserungsvorschlägen.
Die Probanden wurden auch befragt, ob sie sich einen Einsatz von 3D-Megamaps mit MR-HMDs in der realen Welt sowohl in Innen- als auch Außenbereichen vorstellen können.

\section{Testgruppe}
An der Nutzerstudie nahmen 15~Personen teil ($user\_0, \dots, user\_14$), davon 10~männlich und 5~weiblich.
Die Probanden befanden sich in einem Altersbereich von 23 bis 35 Jahren.
Vier der Probanden sind Brillenträger von denen zwei die Brille aus Bequemlichkeitsgründen während des Tests abnahmen.
Zwei der Probanden sprachen Englisch und füllten ins Englische übersetzte Versionen der Fragebögen aus.
Die Muttersprache aller anderen Probanden ist Deutsch.
Keiner der Probanden war in die Inhalte der Masterarbeit oder der Nutzerstudie eingeweiht.

Die Probanden wurden durch an der Universität ausgehängte Flyer (siehe \autoref{appendix:study_material}), den Mailverteiler des Fachbereichs~3 und über die Arbeitsgruppe Human-Computer~Interaction rekrutiert.
Einige Probanden wurden außerdem durch persönlichen Kontakt mit dem Versuchsleiter angeworben.
Für die Teilnahme wurden den Probanden Snacks und Getränke angeboten.
Eine finanzielle Aufwandsentschädigung gab es nicht.

Bei $user_4$ kam es während der ersten Kondition ($3D_h$) zu einem PC-Absturz, der aus einer Inkompatibilität zwischen dem Ryzen-Prozessor und dem VIVE Wireless Adapter resultiert \parencite{HTCCorporation2018c}.
Auch bei einem erneuten Start des Experiments kam es in der Kondition $3D_h$ zum Absturz.
Daraufhin wurde der Test abgebrochen.
Der Proband erschien wenige Tage später und führte das Experiment erneut ohne Absturz durch.
Es wird darauf hingewiesen, dass dieser Nutzer durch die vorige begrenzte Teilnahme für seinen Durchlauf mehr Vorkenntnisse hatte, als die anderen Probanden.

\section{Datenerhebung}
Neben den Antworten aus den Fragebögen wird das Sichtfeld der Proband in der virtuellen Welt aufgezeichnet (und ggf. eine Audioaufnahme des Gesagten).

Zum Vergleich der Performanz von 2D und 3D wird beim Richtungsschätzen die horizontale und vertikale Abweichung von der \enquote{korrekten} Richtung berechnet.
Als korrekte Richtung gilt der Vektor von der Controllerspitze zum Zentrum des Zielraums in der Umgebung.
Die horizontale und vertikale Abweichung ergeben sich aus den jeweiligen Winkeln zwischen der vom Probanden bestätigten Richtung und der korrekten Richtung.

Zum Vergleich der Effizienz werden diverse Zeitmessungen durchgeführt.
Zum einen wird die Zeit gemessen, die Probanden zum Finden des Zielraums benötigen.
Zum anderen wird die Zeit gemessen, bis Probanden die geschätzte Richtung bestätigt haben.
So sollen Aussagen über die Zuversicht der Probanden ihrer Schätzung getroffen werden.

Darüber hinaus werden kontinuierlich die Positionen und Rotationen des HMDs und des Controllers aufgezeichnet.

\section{Ergebnisse}
In den folgenden Abschnitten werden die Ergebnisse aus der Nutzerstudie präsentiert und statistisch analysiert.

\subsection{Santa Barbara Sense-of-Direction Skala}
Vor dem eigentlichen Test füllten Probanden einen Fragebogen nach der SBSOD Skala aus.
Dieser liefert eine Wertung des vom Probanden selbst wahrgenommenen Orientierungssinn.
Zur Auswertung werden die positiv formulierten Elemente invertiert bewertet.
Anschließend wird der Mittelwert über die individuellen Bewertungen berechnet.
Der Mittelwert entspricht der SBSOD-Wertung \parencite{Hegarty2002}.
\autoref{fig:sbsod} stellt das Ergebnis grafisch dar.
Von den 14 Probanden erreichten 12 (\SI{85,71}{\percent}) einen Wert über 3 (\enquote{neutral}).
Im Schnitt wurde eine Bewertung von 3,38 ($\pm$ 0,41) erzielt.
Der höchste erreichte Wert ist 4,07, der niedrigste Wert ist 2,67.
\begin{figure}[h]
    \centering
    \includegraphics[trim={3cm, 0, 2.5cm, 0}, clip, width=\linewidth]{figures/analysis/sbsod}
    \caption{Erzielte SBSOD-Werte mit $\mu = \num{3,38}$ und $\sigma = \num{0,41}$.}
    \label{fig:sbsod}
\end{figure}

\subsection{Effektivität und Effizienz bei Raumsuche}
\subsubsection*{Suchzeit}
\label{ssec:searchtime}

\autoref{fig:searchtimes} zeigt die durchschnittlichen Suchzeiten bis zur Auswahl des Zielraums über alle Karten pro Kondition.
Die entsprechenden Mittelwerte und Standardabweichungen sind in \autoref{tab:searchtime_overview} aufgeführt.
Zwischen den drei Konditionen ist nach dem Friedman-Test ein signifikanter Unterschied festzustellen ($\chi^2(2) = \num{63.5}, p < 0.001$).
Ein paarweiser Wilcoxon-Vorzeichen-Rang-Test zwischen den einzelnen Konditionen zeigt, dass die Suche nach dem Zielraum in $2D$ ($\num{18.31} \pm \SI{9.3}{\second}$) signifikant schneller ist als in $3D_l$ ($\num{27.44} \pm \SI{13.39}{\second}, z=\num{-6.003}, p<\num{0.001}$) und $3D_h$ ($\num{28.84} \pm \SI{10.08}{\second}, z=\num{-7.136}, p<0.001$).
Der Unterschied der Suchzeit zwischen $3D_l$ und $3D_h$ ist jedoch nicht signifikant.
\begin{figure}[h!]
    \centering
    \includegraphics[width=0.7\linewidth]{figures/analysis/searchtime_boxplot}
    \caption{Boxplot der durchschnittlichen Suchzeiten (in Sekunden) über alle Karten je Kondition.}
    \label{fig:searchtimes}
\end{figure}

\begin{table}[h!]
    \centering
    \caption{Mittelwert $\mu$ und Standardabweichung $\sigma$ der Suchzeit über alle Karten je Kondition.}
    \label{tab:searchtime_overview}
    \begin{tabular}{rccc}
        \toprule
        {} & \multicolumn{3}{c}{Suchzeit [\SI{}{\second}]} \\
        {} &          $3D_l$ &          $3D_h$ &          $2D$ \\
        \midrule
        $\mu$    &  $\num{27.44}$     &  $\num{28.84}$     &  $\num{18.31}$ \\
        $\sigma$ &  $\pm \num{13.39}$ &  $\pm \num{10.08}$ &   $\pm \num{9.3}$ \\
        \bottomrule
    \end{tabular}
\end{table}

Weiterhin stellt sich die Frage, ob ein \enquote{besserer} Orientierungssinn zu einer besseren Suchzeit führt.
\autoref{fig:correlation_time_sbsod} zeigt die durchschnittliche Suchzeit der Probanden in Bezug zu den ermittelten SBSOD-Werten.
Für jede Kondition wird Spearmans Rangkorrelationskoeffizient zwischen der SBSOD-Wertung und der durchschnittlichen Suchzeit ermittelt.
Die Tests weisen keinen signifikanten Zusammenhang der Variablen auf.
\begin{figure}[ht]
    \centering
    \includegraphics[width=\linewidth]{figures/analysis/correlation_time_sbsod_scatter}
    \caption{Scatterplot zur Überprüfung der Korrelation zwischen SBSOD Wertung und Suchzeit.}
    \label{fig:correlation_time_sbsod}
\end{figure}

\subsubsection*{Fehlerrate bei Raumauswahl}
Neben der Suchzeit wurde auch die Zahl der falsch ausgewählten Räume (die Nicht-Zielräume) gemessen.
Über die insgesamt 252 Suchaufgaben (14~Probanden $\times$ 3~Konditionen $\times$ 6~Karten) wurden in drei Fällen (\SI{1,19}{\percent}) vor dem Zielraum falsche Räume ausgewählt.
\autoref{tab:error_searching} zeigt eine Übersicht der entsprechenden Events.
Die Fehler treten jeweils in der ersten Kondition der getesteten Sequenz auf (jedoch nicht in der ersten Karte).
\begin{table}
    \centering
    \caption{Übersicht der Anzahl der falsch ausgewählten Räume während der Suchaufgabe.}
    \label{tab:error_searching}
    \begin{tabular}{rccccc}\toprule
        Proband                  & Kondition               & Karte    & Suchzeit [\SI{}{\second}] & \# Fehler & Sequenz  \\\midrule
        user\_8                  & $3D_h$                  & Karte\_6 & $\SI{40,61}{\second}$     & 2         & $3D_h$--$3D_l$--$2D$ \\
        user\_9                  & $2D$                    & Karte\_5 & $\SI{38,4}{\second}$      & 1         & $2D$--$3D_l$--$3D_h$ \\\bottomrule
    \end{tabular}
\end{table}

\subsection{Effektivität und Effizient bei Richtungsschätzung}
\subsubsection*{Genauigkeit der Schätzung}
\autoref{fig:horizontal_offset} zeigt das Ergebnis der Richtungsschätzung.
Da die Abweichung sowohl in positiven Winkeln (Abweichung nach rechts) als auch negativen Winkeln (Abweichung nach links) gemessen wurde, werden hier die absoluten Werte der Abweichung analysiert.
Andernfalls würden die Werte sich bei der Mittelwertbildung gegenseitig aufheben.
Wie zu erkennen ist, liegen die Mittelwerte von $3D_l$~(7,79 $\pm$ \ang{7,22}), $3D_h$~(7,84 $\pm$ \ang{7,44}) und $2D$~(6,62 $\pm$ \ang{5,28}) nahe beieinander.
Der Friedman-Test kann keine signifikanten Unterschiede zwischen den Konditionen nachweisen.
\begin{figure}
    \centering
    \includegraphics[height=0.45\textheight]{figures/analysis/horizontal_offset}
    \caption{Horizontale Abweichung in $\degree$ (Grad) bei der Richtungsschätzung von der tatsächlichen Linie zur Raummitte (ohne vertikale Abweichung).}
    \label{fig:horizontal_offset}
\end{figure}

Für die einzelnen Karten fasst \autoref{tab:herror_per_map} die Mittelwerte und Standardabweichungen der absoluten horizontalen Abweichungen zusammen.
\autoref{fig:horizontal_offset_per_map} zeigt die Unterschiede zwischen den Karten grafisch.
Es ist zu erkennen, dass vor allem die Abweichungen bei Karte 3 und 4 größer sind, als auf den anderen Karten.
Der Friedman-Test bestätigt signifikante Unterschiede zwischen den Karten ($\chi^2(6) = \num{39.755}, p < \num{0.001}$).
Die horizontale Abweichung bei Karte~3 ist signifikant größer als bei Karte~1 ($z = \num{-4,483}, p < \num{0,001}$), Karte~2 ($z = \num{-4,195}, p < \num{0,001}$), Karte~5 ($z = \num{-3,432}, p < \num{0,001}$) und Karte~6 ($z = \num{-3,37}, p < \num{0,001}$).
Auch die hohe Abweichung bei Karte~4 ist gegenüber Karte~1 ($z = \num{-3,782}, p < \num{0,001}$), Karte~2 ($z = \num{-3,395}, p < \num{0,001}$), Karte~5 ($z = \num{-2,044}, p < \num{0,05}$) und Karte~6 ($z = \num{-2,17}, p < \num{0,05}$) signifikant.
Weiterhin gibt es bei Karte~5 eine signifikant größere Abweichung als bei Karte~1 ($z = \num{-2,645}, p < \num{0,008}$), was durch die Ausreißer (siehe \autoref{fig:horizontal_offset_per_map}) begünstigt wird.
\begin{figure}[h!]
    \centering
    \includegraphics[width=0.75\linewidth]{figures/analysis/horizontal_offset_per_map}
    \caption{Absolute horizontale Abweichung in $\degree$ (Grad) bei der Richtungsschätzung pro Karte.}
    \label{fig:horizontal_offset_per_map}
\end{figure}%

\begin{table}[h!]
    \centering
    \caption{Mittelwert $\mu$ und Standardabweichung der absoluten horizontalen Abweichung pro Karte in $\degree$.}
    \label{tab:herror_per_map}
    \begin{tabular}{cccccc}\toprule
        Karte 1 & Karte 2 & Karte 3 & Karte 4 & Karte 5 & Karte 6 \\\midrule
        $\mu = \ang{4,44}$ & $\mu = \ang{4,96}$ & $\mu = \ang{11,75}$ & $\mu = \ang{10,19}$ & $\mu = \ang{6,62}$ & $\mu = \ang{6,55}$ \\
        $\pm \ang{4,03}$ & $\pm \ang{3,3}$ & $\pm \ang{6,98}$ & $\pm \ang{9,07}$ & $\pm \ang{4,94}$ & $\pm \ang{7,05}$ \\\bottomrule
    \end{tabular}
\end{table}

Wird der Unterschied zwischen der positiven und negativen horizontalen Abweichung mit einbezogen, ist eine Tendenz zu einem Ausschlag nach links (negativ) der horizontalen Abweichungen zu erkennen.
Von den insgesamt 252~Richtungsschätzungen weichen 158 (\SI{62,7}{\percent}) nach links ab.

\subsubsection*{Schätzungszeit}
Neben der Abweichung der Schätzung wurde auch die Zeit gemessen, welche die Probanden für die Abgabe der Schätzung benötigen.
\autoref{fig:pointing_time} visualisiert die erhobenen Daten.
In den meisten Iterationen (\SI{95,24}{\percent}) wurde die Richtungsschätzung innerhalb von 10~Sekunden bestätigt.
Im Schnitt benötigten die Probanden zur Schätzung in $3D_l$ $\num{3,61} \pm \SI{1,71}{\second}$, in $3D_h$ \mbox{$\num{4,18} \pm \SI{2,86}{\second}$} und in $2D$ $\num{4,86} \pm \SI{4,78}{\second}$.
Der paarweise Vergleich weist nicht auf statistisch signifikante Unterschiede hin.
Allerdings ist eine Tendenz zu erkennen, dass die Probanden sich in der $2D$ Kondition mehr Zeit gelassen haben als in $3D_l$ ($z = -7.947,	p = 0.083$).
\begin{figure}[h]
    \centering
    \includegraphics[width=\linewidth]{figures/analysis/pointing_time}
    \caption{Vergleich zwischen der mittleren Schätzungszeit der Konditionen, die die Probanden für die Abgabe einer Richtungsschätzung benötigten.}
    \label{fig:pointing_time}
\end{figure}

Einer der Gründe für die Messung der Schätzzeit war es, die Korrelation zur resultierenden horizontalen Abweichung zu testen.
Die Überlegung ist, dass Probanden, die sich mehr Zeit für die Schätzung der Richtung nehmen, präziser sind.
\autoref{fig:pointing_time_vs_error} zeigt die absolute horizontale Abweichung in Bezug zur benötigten Schätzzeit.
Weder grafisch, noch doch Anwendung des Spearman-Tests, ist eine Korrelation zwischen den beiden Variablen erkennbar.
Da die meisten Probanden ihre Schätzung innerhalb der ersten 10~Sekunden abgegeben haben, sind für den Zeitraum \emph{nach} 10~Sekunden zu wenig Messungen vorhanden, um eine zuverlässige Aussage über eine potentielle Korrelation treffen zu können.
\begin{figure}[hb]
    \centering
    \includegraphics[width=\linewidth]{figures/analysis/pointing_time_vs_error}
    \caption{Plot der Schätzungszeit gegen die absolute horizontale Abweichung. %
    Eine Korrelation ist nicht erkennbar.}
    \label{fig:pointing_time_vs_error}
\end{figure}

\subsubsection*{Trefferrate}
Neben der Abweichung wurde auch die Trefferrate (also ob der Zielraum bei der Schätzung getroffen wurde oder nicht) gemessen.
\autoref{fig:hitrate_per_user} zeigt einen Boxplot der Trefferrate über alle Probanden zwischen den einzelnen Konditionen.
Es ist anzumerken, dass für die Berechnung eines Treffers die vertikale Abweichung mit einbezogen wurde.
Die höchste durchschnittliche Trefferrate zeigten die Nutzer in $3D_h$ ($\mu = \SI{79,8}{\percent} \pm \SI{26,3}{\percent}$), gefolgt von $2D$ ($\mu = \SI{76,2}{\percent} \pm \SI{23,3}{\percent}$) und schließlich $3D_l$ ($\mu = \SI{75}{\percent} \pm \SI{20,4}{\percent}$).
Ein Friedman-Test ergab keine signifikanten Unterschiede zwischen den Konditionen.
\begin{figure}[h]
    \centering
    \includegraphics[width=0.7\linewidth]{figures/analysis/hitrate_per_user}
    \caption{Trefferrate des Zielraums über alle Probanden zwischen den Konditionen.}
    \label{fig:hitrate_per_user}
\end{figure}

\subsection{Fragebögen zur Nutzungsfreundlichkeit}
Die Fragebögen enthielten sowohl positiv als auch negativ formulierte Aussagen in einer ungleichmäßigen Reihenfolge, denen die Probanden auf einer Skala von 1 bis 5 zustimmen oder sie ablehnen konnten.
Für die Auswertung wurden die Bewertungen der positiven Aussagen invertiert.
Die Bewertungen wurden aufaddiert und schließlich mit einem Faktor von \num{3,125} multipliziert, um so eine Wertung zwischen 0 und 100 zu erhalten.
Es sei darauf hingewiesen, dass die Fragebögen nur acht statt, wie im \emph{System Usability Scale} vorgesehen, zehn Fragen enthalten \autocite{Brooke2013}. 
Daher sind die Wertungen der beiden Skalen nicht direkt miteinander vergleichbar.

Das Ergebnis des Fragebogens ist in \autoref{fig:usability} zusammengefasst.
Durchschnittlich bewerteten die Probanden die $2D$-Variante am besten ($\mu = \num{83,75} \pm \num{12,11}$).
Dahinter liegen die Megamaps mit $3D_l$ ($\mu = \num{72,29} \pm \num{17,27}$) und $3D_h$ ($\mu = \num{65,21} \pm \num{18.06}$).
Ein paarweiser t-Test zeigt, dass die höhere Bewertung von $2D$ gegenüber $3D_l$ ($t = 2.452, p < 0.05$) und $3D_h$ ($t = 3.988, p < 0.005$) statistisch signifikant ist.
Zwischen den beiden 3D-Konditionen gibt es keine statistisch signifikanten Unterschiede.
\begin{figure}
    \centering
    \includegraphics[width=0.7\linewidth]{figures/analysis/usability}
    \caption{Bewertung der Nutzbarkeit der einzelnen Konditionen.}
    \label{fig:usability}
\end{figure}



\section{Diskussion der Ergebnisse}

\subsection{Suchaufgabe}
Sowohl die Zeitmessungen als auch die Präferenzen der Probanden für die Suchaufgabe ist eindeutig.
Die 2D-Variante ist hier den beiden 3D-Megamaps mit einer durchschnittlichen Differenz von fast \SI{10}{\second} weit überlegen.
Es gibt drei Gründe durch die sich dieses Ergebnis erklären lässt:
\begin{itemize}
    \item
    Für die Suchaufgabe wurden die Bälle in den Räumen platziert.
    Die Bälle befanden sich dabei auf Fußbodenhöhe des jeweiligen Raums.
    Auf den 3D-Megamaps wurden sie dabei teilweise durch die Wände der Räume und die geöffneten Türen verdeckt.
    Die Probanden mussten sich von Raum zu Raum bewegen, um in die Räume zu schauen und die Bälle sehen zu können, was zusätzliche Zeit in Anspruch nahm.
    Aus den Gesprächen ergab sich, dass dieser Effekt auf der hohen Megamap zusätzlich verstärkt wurde.
    % TODO: Zitate
    Die zusätzliche Höhe der Megamap vom Boden sorgte für einen flacheren Blickwinkel auf die Karte, wodurch die Räume schlechter eingesehen werden konnten als auf der niedrigen Megamap.
    Einige Male wurde Bälle nicht gesehen, wodurch die Probanden die Räume mehrmals ablaufen mussten, bis sie den Zielraum entdeckten.
    Auf der 2D-Karte bestand dieser Nachteil nicht, da es in der zweidimensionalen Ansicht in diesem Sinn keine Wände gibt und die Bälle nicht verdeckt werden.
    Die Probanden schauten hier automatisch von oben auf die Karte und konnten so direkt in \emph{alle} Räume schauen.
    % TODO: Zitate
    % TODO: Suchfunktion?
    
    \item
    Verbunden mit dem ersten Problem ist das unterschiedliche Sichtfeld der Karte zwischen den Konditionen.
    Bei der 2D-Karte standen die Probanden mehrere Meter von der Karte entfernt und schauten \enquote{von oben} auf sie hinab.
    Die 2D-Karte war somit zu jeder Zeit komplett im Blickfeld der Probanden, wodurch alle Räume und Bälle sowie der eigene Positionsmarker sichtbar waren.
    Bei den 3D-Megamaps hingegen sieht das Konzept vor, dass die Probanden inmitten der Karte stehen und von ihr umgeben werden.
    Das bedeutet, dass der Teil der Megamap, der sich hinter den Probanden befindet, nicht einsehbar ist.
    Demnach können sich Probanden nur auf einen kleinen Teil der Karte konzentrieren.
    Einige Probanden versuchten dieses Problem zu umgehen, indem sie sich bei jeder Karte zuerst \enquote{aus der Karte heraus} bewegten, um einen Überblick des Grundrisses zu bekommen.
    % TODO: Zitate
    Ein weiteres Problem war, dass sich bei den 3D-Karten der Positionsmarker des Probanden unter Umständen außerhalb des Blickfelds befand.
    Die Probanden verloren somit während der Suche den Bezug zum Laborraum.
    user\_9 beschreibt, dass sie jedes Mal nach dem Lab-Raum suchen musste, weil sie \enquote{voll auf die Bälle fixiert} war.
    Der Verlust von visuellem Kontext außerhalb des Blickfelds wird z.B. in Videospielen durch die Anzeige von Markern auf Kompassleisten oder Minikarten ausgeglichen.
    In der Literatur lassen sich Ansätze finden, die Gleiches für die Anwendung auf AR-VR-HMDs umsetzen \autocites{Lin2017a}{Gruenefeld2017}.
    
    \item
    Die Probanden beschrieben, dass zum Merken der Richtung für die Schätzungsaufgabe auf den Karten unterschiedliche \enquote{Taktiken} notwendig waren.
    In der Tat folgten fast alle Probanden auf den 3D-Megamaps der gleichen Vorgehensweise und kehrten vor der Auswahl des Zielraums zuerst zum Laborraum auf der Karte zurück, um sich so für die Schätzungsaufgabe die Richtung merken zu können.
    Ein Nutzer zielte während des Ablaufens der Karte mit dem Controller immer auf den Raum, in dem er die höchste Zahl an Bällen gezählt hatte ().% TODO Zitat -> Nils?)
    Auf der 2D-Karte hingegen waren diese Taktiken nicht notwendig.
    Hier orientierten sich die Probanden für das Merken der Richtung an besonderen Merkmalen auf der Karte, z.B. der Säule im Laborraum oder den Türen.
    Einige der Probanden nutzen auch die Rotation des Positionsmarkers aus, um sich den Winkel für die korrekte Drehung zum Zielraum zu merken.
    %TODO Zitate
\end{itemize}

Ein Aspekt, der sich (laut den Probanden) weniger auf die Performance auswirkte als zuvor angenommen, war der Komfort der HMD-Nutzung.
Da die Probanden für die 3D-Megamaps die meiste zeit in Richtung Boden schauen mussten, wurde der Kopf in einer geneigten Position gehalten.
Es wurde vorab vermutet, dass das Gewicht des HMDs sowie die Befestigung am Kopf für ein unangenehmes Gefühl sorgen könnte und den Eindruck, das HMD könne vom Kopf abrutschen.
In der Praxis war das nur für einen Probanden problematisch (user\_5).
Die anderen Probanden berichteten von keinen Problemen mit dem Komfort.
Die Tatsache, dass an den Rändern der Linse das Bild (und somit auch die Bälle in den Räumen) verschwommen war, wurde jedoch angemerkt.

\subsection{Richtungsschätzung}
% Pfade merken wäre schön gewesen.

\subsection{Usability}

\subsection{Animation und Room-Guides}

\subsection{Zusammenfassung}

%
\cleardoublepage
