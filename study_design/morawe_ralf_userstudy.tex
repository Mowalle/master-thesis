\documentclass[
    draft=false,
    paper=a4,
    fontsize=11pt,
    twoside=false,
    captions=tableheading,
    british, ngerman,
]{scrartcl}

\usepackage{babel}

\usepackage{fontspec}
\usepackage[final]{microtype}

\usepackage{booktabs}
\usepackage{csquotes}

\usepackage{hyperref}

\begin{document}

\begin{center}
    \LARGE\textsf{\textbf{Plan für User Study}}
\end{center}

\noindent
Digitale Kartenanwendungen wie z.B. \emph{Google Maps} bieten Nutzern eine Vielzahl von Funktionen, um diese bei der Navigation und Kartenexploration unterstützen.
Da diese Anwendungen allerdings auf GPS angewiesen sind, können sie nur im Freien eingesetzt werden.
Kartenanwendungen für Innenräume sind hingegen nur für spezielle Gebäude verfügbar.
Die Forschung befasst sich daher mit den besonderen Anforderung von Indoor-Karten, wie zum Beispiel der Darstellung mehrstöckiger Gebäude.
Dabei wird jedoch meistens nur die Navigation von einem Punkt zu einem anderen als Nutzungsbeispiel behandelt.
Die Kartenexploration von Gebäuden mittels digitaler Karten wir bisher wenig betrachtet.

Im Rahmen dieser Masterarbeit wird untersucht, ob sich eine digitale 3D-Karte für die Anwendung der Kartenexploration eignet.
Zu diesem Zweck wird mit VR-Technologie eine 3D-Karte direkt in der Umgebung des Nutzers platziert.
Dieser kann die 3D-Karte für eine Reihe von Explorationstasks nutzen, die von bereits existierenden Kartenanwendungen inspiriert sind.

Das Ziel dieser User Study ist zu ermitteln, ob eine in die Umgebung integrierte 3D-Karte für die Kartenexploration besser geeignet ist als eine herkömmliche Darstellung in 2D.

\section*{Setup}
Das Experiment wird (vorraussichtlich) im Laborraum 5230 des MZH an der Uni Bremen durchgeführt.
Als Hardware wird ein Desktop-PC inklusive Monitor (für den Versuchsleiter) sowie ein \emph{HTC Vive} HMD inklusive Lighthouses eingesetzt.
Die virtuelle Umgebung, in der der Nutzer sich wiederfindet, ist eine Nachbildung des Laborraums.
Die Ausgangsposition, an die der Nutzer sich zu Beginn stellen soll, ist sowohl in der echten als auch der virtuellen Umgebung mit einem \textbf{X} gekennzeichnet.

\section*{Hypothesen}
\begin{itemize}
    \item Nullhypothese $H_0$: Es gibt keinen signifikant messbaren Unterschied zwischen der Nutzung der Indoor-2D-Karte und der umgebungsintegrierten Indoor-3D-Karte.
    \item $H_1$: Die umgebungsintegrierte 3D-Karte ist für die Indoor-Kartenexploration besser geeignet als die Darstellung in 2D.
    \begin{itemize}
        \item Ziele auf der Karte werden schneller gefunden und ausgewählt.
        \item Die räumliche Vorstellung der Umgebung (insbesondere die Position der Ziele relativ zur eigenen Position) ist präziser.
    \end{itemize}
    \item $H_2$: Die User Experience ist mit der umgebungsintegrierten 3D-Karte besser.
\end{itemize}

\section*{Ablauf}
Das Experiments wird in mehreren Phasen durchgeführt:

\vspace{1em}
\noindent
\textit{Phase 1:}
Der Proband unterschreibt eine Einverständniserklärung bezüglich der Teilnahme an der User Study, mögliche Nebenwirkungen und Risiken bei der Bedienung der VR-Equipments sowie der Aufnahme der benötigten Daten und der Veröffentlichung dieser in anonymisierter Form im Rahmen dieser Arbeit.

\vspace{1em}
\noindent
\textit{Phase 2:}
Diese Phase dient dazu, dass der Proband den Umgang mit VR sowie mit den beiden Kartendarstellungen lernen kann.
Der Proband setzt das HMD auf und betritt die virtuelle Umgebung.
Er finden sich im virtuellen Labor wieder.
Dort werden gleichzeitig beide Kartendarstellungen angezeigt (3D im Raum, 2D an der Wand).
Als Gebäude wird ein mehrstöckiges Testgebäude verwendet, welches in den späteren Phasen nicht weiter eingesetzt wird, um einen Lerneffekt zu verhindern.
Der Proband kann nun frei Zielpunkte suchen und die Stockwerke wechseln.
Diese Phase dauert 10 Minuten.

\vspace{1em}
\noindent
\textit{Phase 3:}
Der Proband wird einer von zwei Gruppen zugeteilt (siehe \autoref{tab:sequences}): entweder er testet zuerst die 2D- und dann die 3D-Sequenz, oder umgekehrt.
Die Sequenzen beinhalten jeweils eine feste Reihenfolge von Tasks (siehe unten) und anschließend einen Fragebogen.

\begin{table}[h]
    \label{tab:sequences}
    \centering
    \caption{Testsequenzen}
    \begin{tabular}{ccccc}
                  & \multicolumn{2}{c}{Sequenz 1} & \multicolumn{2}{c}{Sequenz 2} \\\toprule
        Gruppe 1: & 2D & Fragebogen & 3D & Fragebogen \\
        Gruppe 2: & 3D & Fragebogen & 2D & Fragebogen
    \end{tabular}
\end{table}

\section*{Konditionen}
Für diese Studie werden zwei Konditionen miteinander verglichen: 2D und umgebungsintegriertes 3D.
\begin{itemize}
    \item Für die 2D-Karte wird eine 2D-Indoor-Karte an einer der Wände des virtuellen Labors platziert.
    \item Die 3D-Karte wird so im virtuellen Labor platziert, dass sich der Nutzer um diese herum bewegen kann.
\end{itemize}
Damit die Konditionen vergleichbar bleiben bieten beide Darstellungen die gleichen Interaktionen an.
Es wird nur ein Stockwerk zur Zeit angezeigt.
Zielpunkte (Points of Interest) zeigen an, ob sie sich oberhalb oder unterhalb des aktuellen Stockwerks oder darauf befinden.
Eine Änderung des Stockwerks sowie die Auswahl von Zielpunkten ist über grafische Elemente möglich.

\section*{Tasks}
Die Tasks laufen alle nach dem gleichen Prinzip ab.
Zuerst sucht der Proband nach Vorgabe einen Zielpunkt auf der Karte.
Wenn er diesen gefunden hat, wählt er den Punkt aus und bestätigt die Auswahl.
Danach wird die Karte ausgeblendet.
Nun muss der Proband mit dem Vive-Controller in die Richtung zeigen, in der sich der gewählte Zielpunkt in der virtuellen Umgebung befindet.
Die Richtung wird durch einen Strahl symbolisiert.
Der Proband bestätigt die Richtung durch Betätigen eines Knopfes und kann die Auswahl danach wiederholen, um Fehler zu korrigieren.

Es werden drei Arten von Suche als Aufgabe gestellt:
Bei der \emph{spezifischen Suche} sucht der Proband einen einzigartigen Punkt auf der Karte (z.B. \enquote{Finde das Büro von Person XY}).
Bei der \emph{semi-offenen Suche} sucht der Proband ebenfalls einen speziellen Punkt, muss aber zwischen mehreren Ähnlichen auswählen (z.B. \enquote{Finde die \textbf{nächstgelegene} Toilette}).
Bei der \emph{offenen Suche} kann der Proband frei innerhalb einer Kategorie wählen (z.B. \enquote{Finde einen Drucker}).

\section*{Metriken und Datenerhebung}
Zum Vergleich von 2D und 3D werden die Zeiten zur Auswahl der Zielpunkte und Fehlerrate gemessen.
Außerdem wird der Abweichungsfehler bei der Richtungsangabe gemessen, aus dem Aussagen über das räumliche Verständnis durch Nutzung der Karte abgeleitet werden sollen.

Da die 3D-Karte eine flexible Ansicht aus allen Richtungen erlaubt soll auch eine Heatmap der Position und Blickrichtung ermittelt werden.

Das Videobild der Vive wird zur Dokumentation ebenso aufgenommen.

Zur Erhebung der qualtiativen Daten über die User Experience der 3D-Karte werden Fragebögen verwendet, die nach entsprechenden Standards ausgewertet werden können.

\end{document}
