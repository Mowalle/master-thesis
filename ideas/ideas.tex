\documentclass[a4paper]{scrartcl}

% Language
\usepackage[british, ngerman]{babel}

% Font stuff and typesetting stuff
\usepackage[utf8]{inputenc}
\usepackage[T1]{fontenc}
\usepackage{microtype}

% Font
\usepackage{lmodern}

% Other
\usepackage{amsmath, amssymb}
\usepackage{csquotes}
\usepackage{float}
\usepackage{graphicx}
\usepackage{lipsum}
\usepackage[obeyDraft]{todonotes}

\usepackage[colorlinks]{hyperref}

\usepackage[
	backend = biber, % Sets biber as backend.
	style = authoryear-icomp, % Use [author, year] citing with ibid. when duplicate.
	maxnames = 3, % Truncate lists above 3 entries (e.g. names). Both citing and bib.
	minbibnames = 3, % Truncate to 3 entries (e.g. names). Both citing and bib.
	mincitenames = 1,
	%backref = true,
]{biblatex}
\addbibresource{references.bib}

\setkomafont{captionlabel}{%
	\bfseries
}

% Dank Koma-Klasse
\renewcaptionname{ngerman}{\subsectionautorefname}{Abschnitt}
\renewcaptionname{ngerman}{\figureautorefname}{Abb.}
\renewcaptionname{ngerman}{\tableautorefname}{Tab.}

\usepackage[
	automark,
	headsepline,
%	headtopline,
%	footsepline
]{scrlayer-scrpage}
%\pagestyle{scrheadings}
\ihead{Ralf Morawe}
\ohead[\pagemark]{\pagemark}
\chead{Ideen für Masterarbeit}
\cfoot[]{}

\begin{document}

\section*{Idee 1: Augmented Navigation}

Heutzutage verwenden Menschen digitale Navigationssysteme, um von einem Ort zum nächsten zu gelangen.
Bei diesen Systemen handelt es sich um GPS-Geräte oder auch mobile Apps.
Allerdings sind diese Systeme nicht ohne Fehler.
\textcite{Lin2017} analysieren das Phänomen \enquote{Death by GPS}.
Dieses beschreibt negative Ereignisse, die in Folge von der Nutzung solches Systeme auftreten.
Dazu zählen sowohl Fehlinformationen durch das System sowie schwierige Bedienung durch die Nutzer.
Ein Ansatz um diese Navigation einfacher und sicherer zu machen ist, die bisher meist iterative Navigation durch die Referenzierung von Landzeichen als Orientierungshilfe zu erweitern \parencites{Fingas2018}{Wenig2017}.

\begin{figure}[h]
	\centering
	\includegraphics[width=0.6\textwidth]{figures/SR4GPS.png}
	\caption{In-Game GPS im Computerspiel \emph{Saints Row IV}. Der wegweisende Pfeil \enquote{schwebt} auf der Straße und zeigt die Richtung an.}
	\label{fig:sr4_gps}
\end{figure}

Auch in Spielen nimmt Navigation eine wichtige Rolle ein.
Ein interessanter Ansatz wird im Spiel \emph{Saints Row IV} präsentiert (siehe \autoref{fig:sr4_gps}).
Wenn Spieler einen Wegpunkt für ihre Ziele setzen, wird nicht nur auf einer Karte die kürzeste Route markiert.
Es werden zusätzliche Wegweiser an wichtigen Punkten der Straßenführung \emph{im direkten Sichtfeld} der Spieler angezeigt, wodurch die Navigation um einiges vereinfacht wird.

Dieses Prinzip kann dank neuer Augmented-Reality-Technologien (zum Beispiel die \emph{HTC Vive} oder die \emph{Magic Leap} auf die Navigation in der realen Welt übertragen werden.
Das Ziel dieser Arbeit wäre es, eine landzeichen-basierte Navigation mithilfe dieser AR-Headsets zu implementieren.
Die Navigationshinweise würden dann (anstatt nur in 2D und/oder textueller Form) als Objekte im Sichtfeld der Nutzer angezeigt werden.
Von besonderem Interesse sind hier Einblendungen von Wegweisern sowie die Hervorhebung von Landzeichen.
Eine weitere Herausforderung könnte das Rendern dieser Objekte ohne Verdeckungen (\emph{Occlusion}) darstellen, was die Navigation negativ beeinflussen könnte.
Während \textcite{Walton2017} ein bildbasiertes Verfahren zur Verbesserung der Echtzeit-Verdeckungsberechnung vorschlagen, stellen \textcite{Kasperi2017} ein Verfahren vor, dass Umgebungsinformationen zur Rekonstruktion virtueller Gebäude und damit für die Berechnung verwendet.
Für diese Zwecke könnte auch die \emph{Google Maps API} genutzt werden \parencite{Google2018}.

%\begin{itemize}
%	\item \parencites[35--37]{vive2017}[88--120]{vive2017}[23]{vive2017}
%\end{itemize}

\section*{Idee 2: \dots}

\lipsum[3-4]

\section*{Idee 3: \dots}

\lipsum[5-6]

\printbibliography[nottype=online]
\printbibliography[title={Online Referenzen}, type=online]

\end{document}
